\section{STL}
\subsection{I/O}
\begin{minted}[frame=single,framesep=3pt,tabsize=2,breaklines=true,linenos]{c++}
std::ios::sync_with_stdio(false); //speedup IO (dont combine with printf/scanf)
std::cin.ignore(n); //ignore n characters before continue
std::getline(std::cin, s); //reads whole line into string
std::cin >> std::noskipws; //dont skip whitespace
std::cout << std::fixed; //use fixed-point notation
std::cout << std::setprecision(n); //set the precision to n decimals
std::cout << std::setw(n); //set the length each output will contain
std::cout << std::setfill(c); //set the char to fill the remaining chars from above
\end{minted}
\subsection{String}
\begin{minted}[frame=single,framesep=3pt,tabsize=2,breaklines=true,linenos]{c++}
//read a line
std::string s;
std::getline(std::cin, s);
//string split
std::vector<std::string> split(const std::string &s, char delim) {
    std::vector<std::string> elems;
    std::stringstream ss(s);
    std::string item;
    while (std::getline(ss, item, delim)) {
        elems.push_back(item);
    }
    return elems;
}
//substring
str.substr(pos, LENGTH); //till end if LENGTH is empty
//string find
str.find("substring") //return std::string::npos if not found, else position first letter
\end{minted}

\subsection{Containers}
\begin{minted}[frame=single,framesep=3pt,tabsize=2,breaklines=true,linenos]{c++}
std::vector<int> vec;
//sort from small to large
bool cmp(int a, int b){
    return a < b;
}
std::sort(vec.begin(), vec.end(), cmp); 
std::sort(vec.begin(), vec.end(), std::less<int>()); //use std::greater<int>() for large to small and std::less<int>() for small to large
std::find(vec.begin(), vec.end(), NEEDLE); //use .find() if avaible (map/set)
std::count(vec.begin(), vec.end(), NEEDLE); //use .find() if available (map/set)
//find union/intersection/difference -- both need to be SORTED already
std::vector<int> tmp(vec.size()+vec2.size());
std::set_intersection(vec.begin(), vec.end(), vec2.begin(), vec2.end(), tmp.begin());
std::set_union(vec.begin(), vec.end(), vec2.begin(), vec2.end(), tmp.begin());
std::set_difference(vec.begin(), vec.end(), vec2.begin(), vec2.end(). tmp.begin());
//filter unique elements
auto last = std::unique(vec.begin(), vec.end());
vec.erase(last, v.end());
\end{minted}

\subsection{String streams}
\begin{minted}[frame=single,framesep=3pt,tabsize=2,breaklines=true,linenos]{c++}
//c-style for complex parsing
int in;
sscanf(str.c_str(),"%d",&in)
//convert number to string -- itoa()
std::ostringstream os;
os << (int) i;
os.str();
//convert string to number(s) -- atoi() or atol()
std::istringstream is(str);
is >> i;
\end{minted}

\subsection{Constants}
\begin{minted}[frame=single,framesep=3pt,tabsize=2,breaklines=true,linenos]{c++}
INT_MIN
INT_MAX
LLONG_MIN
LLONG_MAX
PI //defined in header
EPS //defined in header
\end{minted}

\subsection{Math}
\begin{minted}[frame=single,framesep=3pt,tabsize=2,breaklines=true,linenos]{c++}
ceil(a); //round up
floor(a); //round down
round(a); //round nearest
atan2(a, b); //atan with two parameters
frac_part = modf(d, intpart); //split up in parts (int_part is a double!)
std::__gcd(a, b); //greatest common divisor a, b
\end{minted}

\subsection{Permutations}
\begin{minted}[frame=single,framesep=3pt,tabsize=2,breaklines=true,linenos,label=O($N!$)]{c++}
std::vector arr;
std::sort(arr.begin(), arr.end());
do {
   //work with permutation
} while (next_permutation(arr.begin(),arr.end));
\end{minted}

\subsection{Binary search}
\begin{minted}[frame=single,framesep=3pt,tabsize=2,breaklines=true,linenos,label=O(log(N))]{c++}
std::multimap<int, int> m;
m.lower_bound(i); //first element equal or higher then i
m.higher_bound(i); //first element higher then i
std::pair<auto iter, auto iter> p = m.equal_range(i); //get all with equal value
\end{minted} 

\subsection{Bit twiddling hacks}
\begin{minted}[frame=single,framesep=3pt,tabsize=2,breaklines=true,linenos]{c++}
in |= (1 << a); //enable bit at position a
in ^= (1 << a); //toggle bit at position a
if(in & (1 << a)); //check bit at position a
//get last bit set (least signifcant)
c = (b & -b) 
//ll for the long long versions
__builtin_popcount(in); //count the amount of bits set
__builtin_ffs(in); //give the least signifcant index + 1 in binary representation
__builtin_clz(in); //returns the number of leading zeros
__builtin_ctz; //give number of trailing zeros
\end{minted}